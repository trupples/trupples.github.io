\documentclass{article}
\usepackage[a4paper]{geometry}
\usepackage[utf8]{inputenc}
\usepackage{multirow}
\usepackage{tikz}
\usepackage{pgfplots}

\renewcommand{\defaultscriptratio}{1.5}

\title{Calcularea constantei gravitaționale din mișcarea pendulurilor gravitaționale}
\author{Dragomir Ioan, Prelipcean Marius \\ 11B
	\and Constantineanu Raluca \\ Prof. Coordonator
	\and Cristian, Bolea, Turcilă}
\date{}

\begin{document}
\maketitle

\section{Sinopsis}
Folosind date experimentale măsurate de mai multe grupe de elevi în cadrul
orei de fizică și noțiuni te\-o\-re\-ti\-ce clasice despre gravitație, forța
elastică și pendulul gravitațional construim un model ce aproximează
mișcările unui pendul cu un oscilator liniar armonic pe care îl folosim
pentru a aproxima constanta gravitațională din zona Bucureștiului. În final
arătăm și că metoda de calcul prin regresie liniară este mult mai precisă
decât luarea mediei aritmetice a constantelor gravitaționale rezultate din
mai multe experimente independente.

\section{Introducere}

Forța gravitațională este aproximată pe Pământ ca

\begin{math}
	G = mg
\end{math}

unde $G$ este mărimea forței gravitaționale, $m$ este masa obiectului atras
de pământ și $g$ este constanta gravitațională.

Scopul acestui studiu este deducerea valorii constantei gravitaționale din
măsurători experimentale.

Folosind niște greutăți de cântar și niște sfori, grupele de elevi au construit
penduluri gravitaționale. Cum oscilația unui pendul gravitațional la unghiuri
mici\footnote{Se consideră "unghiuri mici" unghiurile $ <5^\circ $}
e aproximată de un oscilator liniar armonic, am calculat modelul OLA-ului
teoretic corespunzător fiecărui pendul construit și am folosit analogia OLA --- pendul
pentru a extrage o relație între datele măsurate în experimente și constanta
gravitațională, astfel calculând-o.

După calcularea unei valori aproximale pentru $g$ se vor discuta posibile surse
de erori în acest proces și soluții pentru creșterea preciziei răspunsului.

\section{Obținerea datelor experimentale}
Fiecare grupă a primit de la profesor câte o sfoară de iută cu lungimi
între 20-50cm (not. $l$) și câte o greutate cu mase între 200-500g (not. $m$). Acestea au
fost folosite pentru a construi un pendul gravitațional a cărui masă și
lungime sunt cunoscute.

După construirea pendulelor, grupele au efectuat mai multe teste în care
lăsau pendulul să oscileze cu un unghi maxim "mic" și măsurau:
1) durata totală a oscilațiilor (not. $ t $); 2) numărul de oscilații (not.
$ n $). Din aceste date se poate calcula pentru fiecare test: 3) perioada
medie a unei oscilații (not. $ T $); și pentru toate testele: 4) media
perioadelor medii ale oscilațiilor (not. $\overline{T}$); 5) eroarea
perioadei medii a fiecărui test (not. $\Delta T$); 6) eroarea medie peste
toate testele (not. $\overline{\Delta T}$).

Pentru a face acest studiu am folosit datele experimentale de la grupele
conduse de Dragomir Ioan (notată G1) și Prelipcean Marius (notată G2). În
Tabelul \ref{table:dateexp} (pag. \pageref{table:dateexp}) apar cele 12
teste înregistrate de cele două grupe, variabilele de pe coloane având
semnificația din paragraful trecut.

\renewcommand{\arraystretch}{1.25}
\begin{table}
\centering
\begin{tabular}{| c | c | c | c | c | c | c | c |}
	\hline
	Gr. & $l$ & $t$ & $n$ & $T={t \over n}$ & $\overline{T}$ & $\Delta T =|T - \overline{T}|$ & $\overline{\Delta T}$ \\
	\hline
	\multicolumn{1}{|c|}{\multirow{5}{*}{G1}} & \multirow{2}{*}{33.5cm} & 27.735s & 20 & 1.136s & \multirow{2}{*}{1.137s} & 0.001s & \multirow{2}{*}{0.001s} \\
	\multicolumn{1}{|c|}{}                    &                         & 27.77s  & 20 & 1.138s &                         & 0.001s & \multicolumn{1}{|c|}{} \\
	\cline{2-8}
        \multicolumn{1}{|c|}{}			  & \multirow{3}{*}{25cm}   & 9.57s   & 10 & 0.957s & \multirow{3}{*}{0.972s} & 0.0155s & \multirow{3}{*}{0.015s} \\
	\multicolumn{1}{|c|}{}                    &                         & 14.72s  & 15 & 0.981s &                         & 0.0095s & \multicolumn{1}{|c|}{} \\
	\multicolumn{1}{|c|}{}	                  &                         & 19.57s  & 20 & 0.978s  &                         & 0.006s & \multicolumn{1}{|c|}{} \\
	\hline
	\multicolumn{1}{|c|}{\multirow{7}{*}{G2}} & \multirow{7}{*}{40cm}   & 11.21s &  9 & 1.245s & \multirow{7}{*}{1.242s} & 0.003s & \multirow{7}{*}{0.007s} \\
	\multicolumn{1}{|c|}{}                    &                         & 12.40s & 10 & 1.240s &                         & 0.002s & \multicolumn{1}{|c|}{} \\
	\multicolumn{1}{|c|}{}			  &                         & 14.80s & 12 & 1.238s &                         & 0.004s & \multicolumn{1}{|c|}{} \\
	\multicolumn{1}{|c|}{}                    &                         & 18.69s & 15 & 1.246s &                         & 0.004s & \multicolumn{1}{|c|}{} \\
	\multicolumn{1}{|c|}{}	                  &                         & 24.96s & 20 & 1.248s &                         & 0.006s & \multicolumn{1}{|c|}{} \\
	\multicolumn{1}{|c|}{}	                  &                         & 31.35s & 25 & 1.254s &                         & 0.012s & \multicolumn{1}{|c|}{} \\
	\multicolumn{1}{|c|}{}	                  &                         & 36.65s & 30 & 1.220s &                         & 0.022s & \multicolumn{1}{|c|}{} \\
	\hline
\end{tabular}

\caption{Rezultate experimentale de la grupele G1, G2}
\label{table:dateexp}
\end{table}

\section{Aproximarea unui pendul cu un OLA}

\begin{figure}
\begin{tikzpicture}
	\draw[dashed] (0,10cm) node[](suport){} -- (0,0);
	\draw (0,10.5cm) node[](){A};
	\draw (suport) -- (5cm, 1.33cm) node[](obj){};
	\draw (2.7cm, 5.66cm) node[](){$l$};
	\draw (0, 9cm) arc[radius=1cm, start angle = 270, end angle = 300] (0.5cm, 9.134cm) node()[pos=0.5,below]{$\alpha$};
	\draw[dotted] (obj) -- (0, 1.33cm) node[](objproj){};
	\draw (obj) node[align=right,above](){B};
	\draw (objproj)+(-0.3cm,0) node[align=left](){B'};
	\draw[dotted] (suport)+(230:10cm) arc[radius=10cm, start angle = 230, end angle = 310];
	\draw (0, 0) node[below](){O};
	\draw[thick,->] (5cm, 1.33cm) -- +(0,-3cm) node[below](){G};
	\draw[thick,->] (5cm, 1.33cm) -- +(1.296cm, -2.25cm) node[above,align=right](){$G_t$};
	\draw[thick,->] (5cm, 1.33cm) -- +(-1.3cm, -0.75cm) node[below,align=right](){$G_n$};
	\draw[thick,->] (5cm, 1.33cm) -- +(-1.3cm, 0) node[above,align=right](){$G_n'$};
\end{tikzpicture}

\caption{Pendul gravitațional}
\label{figure:pendulumola}
\end{figure}

În Figura \ref{figure:pendulumola} (pag. \pageref{figure:pendulumola}) sunt notate:

\hfill \\
$ A $ = punctul de susținere al pendulului \\
$ B $ = capătul liber al pendulului de masă $m$ \\
$ G $ = greutatea \\
$ G_t $ = greutatea tangențială (componenta perpendiculară pe fir) \\
$ G_n $ = greutatea normală (componenta pe direcția firului) \\
$ O $ = poziția de echilibru \\
$ \alpha $ = unghiul dintre poziția curentă și cea de echilibru \\
$l$ = lungimea pendulului \\

Și notăm $B'$ proiecția lui B pe $AO$ și $B_x$ lungimea segmentului $[B'B]$.

Cum capătul liber se poate mișca doar tangențial cercului de rază $l$,
$G_t$ nu efectuează deloc lucru mecanic și toată mișcarea e dictată doar
de $G_n$, variabilă a cărei valoare este:

\begin{math}
	G_n = mg\sin{\alpha}
\end{math}

Datorită restricției că $\alpha < 5^\circ$, putem folosi aproximările:

\begin{center}
	\[	\sin{\alpha} \approx \alpha	\]
	\[	\cos{\alpha} \approx 1		\]
	\[ B_x = l\sin{\alpha} \approx l\alpha	\]
	\[ G_n \approx mg\alpha \approx mg\frac{B_x}{l} \]
\end{center}

Ultimul pas pentru a "găsi" OLA-ul ce aproximează mișcarea pendulului este să
considerăm doar mișcarea pe axa orizontală, deoarece la unghiuri mici mișcarea
pe axa verticală este nesemnificativă:

\begin{center}
	\[ G_n' \approx G_n \cos{\alpha} \approx mg\frac{B_x}{l}\cos{\alpha} \]
\end{center}

Formula aceasta în care forța care acționează asupra unui corp ($G_n'$) depinde
de poziția corpului ($B_x$) este analogul formulei forței elastice din Legea lui
Hooke:
\begin{center}
        \[ F_e = -k\Delta l \]
\end{center}

În acest caz, putem deduce constanta elastică:

\begin{center}
        \[ k = \frac{mg}{l} \]
\end{center}

Acum cunoaștem parametrii unui OLA care imită cât se poate de bine oscilațiile
pendulului. Perioada unui OLA cu $k$ cunoscut este:

\begin{center}
	\[ T = \frac{2\pi}{\sqrt{\frac{k}{m}}} \]
\end{center}

\section{Calcularea constantei gravitaționale}

Care în acest caz devine:
\begin{center}
	\[ T = \frac{2\pi}{\sqrt{\frac{\frac{mg}{l}}{m}}} = \frac{2\pi}{\sqrt{\frac{g}{l}}} = 2\pi\sqrt{\frac{l}{g}} \]
\end{center}

Această formulă demonstrează faptul că perioada de oscilație a unui pendul
e independentă de masa acestuia, însă, mai mult, ne dă o relație clară între
perioada oscilațiilor, lungimea pendulului și constanta gravitațională.

O posibilitate de a proceda în acest stadiu este să se calculeze $g$ pentru
fiecare test în parte și să se considere media aritmetică a $g$-urilor rezultante
ca valoarea finală, însă o metodă mai precisă este estimarea constantei gravitaționale
prin regresie liniară pe perechi $(T^2, l)$.

\begin{center}
	\[ T^2 = 4\pi^2\frac{l}{g} \]
\end{center}

Astfel luăm fiecare din cele 12 teste și le calculăm perechea $(T^2, l)$
(Tabelul \ref{table:tuples} pag. \pageref{table:tuples}), apoi desenăm graficul
acelor perechi (Figura \ref{figure:tuples} pag. \pageref{figure:tuples}).
\begin{figure}[h!]
\begin{minipage}[b]{0.4\linewidth}
	\begin{tabular}{|c|c|c|}
		\hline $T$ & $T^2 \over s^2$ & $l$ \\
		\hline 1.136s & 1.2905 & 0.335m \\
		\hline 1.138s & 1.2950 & 0.335m \\
		\hline 0.957s & 0.9158 & 0.25m \\
		\hline 0.981s & 0.9623 & 0.25m \\
		\hline 0.978s & 0.9564 & 0.25m \\
		\hline 1.245s & 1.5500 & 0.4m \\
		\hline 1.240s & 1.5376 & 0.4m \\
		\hline 1.238s & 1.5326 & 0.4m \\
		\hline 1.246s & 1.5525 & 0.4m \\
		\hline 1.248s & 1.5575 & 0.4m \\
		\hline 1.254s & 1.5725 & 0.4m \\
		\hline 1.220s & 1.4884 & 0.4m \\
		\hline
	\end{tabular}
	
	\caption{Perechile $(T^2, l)$ de la fiecare test}
	\label{table:tuples}
\end{minipage}
\begin{minipage}[b]{0.4\linewidth}
\begin{tikzpicture}
\begin{axis}[%
scatter/classes={%
    a={mark=o,draw=black}}]
\addplot[scatter,only marks,%
    scatter src=explicit symbolic]%
table[meta=label] {
$T2$ l label
	%0 0 a
1.2905 0.335 a
1.2950 0.335 a
0.9158 0.25 a
0.9623 0.25 a
0.9564 0.25 a
1.5500 0.4 a
1.5376 0.4 a
1.5326 0.4 a
1.5525 0.4 a
1.5575 0.4 a
1.5725 0.4 a
1.4884 0.4 a
};
	\draw (axis cs:0,0.01410) -- (axis cs:1.6,0.4141);
\end{axis}	
\end{tikzpicture}
\caption{Graficul perechilor $(T^2, l)$}
\label{figure:tuples}
\end{minipage}
\end{figure}

Dreapta care aproximează cel mai aproape perechile de $(T^2, l)$ culese
este $l = 0.25 T^2 + 0.01410$, deci $g$ rezultat este:

\begin{center}
	\[ g = 4\pi^2\frac{l}{T^2} \approx 39.478\cdot0.25 \]
	\[ g \approx 9.86 \]
\end{center}

\section{Comparație cu metoda naivă}

Acest rezultat este mai precis decât rezultatele obținute independent de
fiecare grupă prin metoda în care se calcula g pentru fiecare test și se ia
media aritmetică a acelor rezultate. Rezultatul după regresia liniară este la
doar 0.5\% de valoarea corectă de 9.81, pe când grupelor le-au dat rezultatele 10.89 (11\% distanța de la valoarea corectă) și 10.43 (4\% distanța).

\section{Surse de eroare}

Deși rezultatul obținut este foarte satisfăcător, acesta e tot destul de departe
de valoarea reală. În această secțiune vor fi analizate mai multe motive pentru
care acesta este cazul și voi propune soluții pentru a obține rezultate și mai
precise în experimentele din viitor.

În primul rând, datorită lipsei de unelte de precizie, măsurătorile au fost făcute
"dupa ochi". Lungimile firelor au fost aproximate în funcție de dimensiunile telefoanelor
și caiatelor și cronometrarea a suferit mult datorită reflexelor umane încete.

În al doilea rând, modelul OLA-ului fără frecare e departe de realitate. Pe de o parte
apare frecarea cu aerul care duce la perioade de oscilație medii mai lungi, deci la
rezultate pentru constanta gravitațională mai mici, iar pe de alta există un defect
la nivel conceptual deoarece un OLA nu poate imita perfect un pendul, ci doar îl poate
aproxima. Oricărui studiu care se bazează pe o astfel de aproximare îi este predestinată
eroarea datorită diferenței subtile dinte un pendul și un OLA, chiar și într-un mediu fără frecare.

Pentru a rezolva prima problemă sugerez dotarea cu unelte de precizie pentru măsurarea
lungimii sforii (rigle cu milimetri) și o soluție automată, posibil electronică, pentru măsurarea
timpului de oscilație. Pentru a doua problemă soluția ar fi schimbarea totală a metodologiei
experimentului, folosirea unor formule mai complexe pentru mișcarea pendulului pentru a evita
aproximările de genul $sin(x) \approx x$.


\section{Concluzie}
În concluzie, calcularea constantei gravitaționale locale este un experiment plauzibil, ușor de efectuat și fără
echipamente de precizie și, în opinia mea, distractiv, la care dacă sunt folosite metode bune de calcul precum regresia liniară în locul
altor soluții mai naive se poate ajunge la rezultate foarte precise. Experimentul are un mare dezavantaj, insă, 
deoarece acesta se bazează pe aproximarea unui pendul cu un OLA. Această aproximare e destul de bună incât să se
poată ajungă la reultate satisfăcătoare, insă nu este decât o aproximare.

\end{document}

